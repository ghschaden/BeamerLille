\documentclass[svgnames,12pt,aspectratio=149]{beamer}
\mode<presentation>%
{
  \usetheme{Boadilla}
  \setbeamercovered{dynamic}
}
\usepackage[english,french]{babel}
\usepackage[utf8]{inputenc}
\usepackage[T1]{fontenc}
\usepackage{BeamerLille}

\usepackage{graphicx}
\usepackage{tikz}


\usepackage{amssymb,amsfonts}

\author{Prénom Nom}
\title{Présentation}
\subtitle{Selon les couleurs de l'Université de Lille}



\begin{document}

\begin{frame}[plain]
  \titlepage
\end{frame}

\section{Introduction générale}
\begin{frame}
  \frametitle{Introduction}

  \begin{itemize}
  \item Point A
  \item Point B
  \item Point C
  \end{itemize}
\end{frame}

\begin{frame}
  \frametitle{Une énumération}

  \begin{enumerate}
  \item Premier
  \item Deuxième
  \item Troisième
  \end{enumerate}
\end{frame}

\begin{frame}
  \frametitle{Environnements}
  \begin{block}{Test: Block}
    Ceci est un block
  \end{block}


  \begin{alertblock}{Test: Alertblock}
    Ceci est un block
  \end{alertblock}

  \begin{exampleblock}{Test: Exampleblock}
    Ceci est un block
  \end{exampleblock}

  Comme vous voyez, les \alert{blocks} et \alert{alertblocks} sont très similaires.
  
\end{frame}

\end{document}


\end{document}

%%% Local Variables:
%%% mode: latex
%%% TeX-master: t
%%% End:
