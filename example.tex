\documentclass[svgnames,12pt,aspectratio=149]{beamer}
\mode<presentation>%
{
  \usetheme{Boadilla}
  \setbeamercovered{dynamic}
}
\usepackage[english,french]{babel}
\usepackage[utf8]{inputenc}
\usepackage[T1]{fontenc}
\usepackage{sty/BeamerLille}

\usepackage{graphicx}
\graphicspath{ {./images/} }
\usepackage{tikz}


\usepackage{amssymb,amsfonts}

\author{Prénom Nom}
\title{Présentation}
\subtitle{Selon les couleurs de l'Université de Lille}



\begin{document}

\begin{frame}[plain]
  \titlepage
\end{frame}

\section{Introduction générale}
\begin{frame}
  \frametitle{Introduction}

  \begin{itemize}
  \item Point A
  \item Point B
  \item Point C
  \end{itemize}
\end{frame}

\begin{frame}
  \frametitle{Une énumération}

  \begin{enumerate}
  \item Premier
  \item Deuxième
  \item Troisième
  \end{enumerate}
\end{frame}

\begin{frame}
  \frametitle{Colonnes}
  Cette partie du diapo s'étend sur toute la largeur du papier, mais on peut aussi créer des colonnes qui n'utilisent qu'une certaine partie

  \begin{columns}
    \begin{column}{0.5\textwidth}
      \begin{itemize}
      \item Première 
      \item Colonne
      \end{itemize}
    \end{column}
    \begin{column}{0.5\textwidth}
      \begin{itemize}
      \item Deuxième
      \item Colonne
      \end{itemize}
    \end{column}
  \end{columns}

  \vspace{1cm}
    On peut faire autant de colonnes qu'on veut, et on peut inclure des images:

    
  \begin{columns}
    \begin{column}{0.33\textwidth}
      \includegraphics[width=1.0\textwidth]{logo-UdLille}
    \end{column}
    \begin{column}{0.33\textwidth}
      \includegraphics[width=1.0\textwidth]{logo-UdLille}
    \end{column}

    \begin{column}{0.33\textwidth}
      \includegraphics[width=1.0\textwidth]{logo-UdLille}
    \end{column}
  \end{columns}
\end{frame}


\begin{frame}
  \frametitle{Environnements}
  \begin{block}{Test: Block}
    Ceci est un block
  \end{block}


  \begin{alertblock}{Test: Alertblock}
    Ceci est un block
  \end{alertblock}

  \begin{exampleblock}{Test: Exampleblock}
    Ceci est un block
  \end{exampleblock}

  Comme vous voyez, les \alert{blocks} et \alert{alertblocks} sont très similaires.
  
\end{frame}

\end{document}


\end{document}

%%% Local Variables:
%%% mode: latex
%%% TeX-master: t
%%% End:
